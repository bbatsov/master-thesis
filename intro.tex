\chapter*{Увод}
\addcontentsline{toc}{chapter}{Увод}
Една система за изучаване на чужди езици трябва да включва в себе си
няколко неща - речников модул, изпитни модули, проверка на правопис,
образователни игри. Има много програми "`речници"' - те имат в себе си
само речников модул и въпреки, че имат своето приложение - те са с
ограничени възможности. За българския пазар същестуват няколко
свободни такива приложения, както и няколко комерсиални. Повечето от
тях споделят следните два проблема - не са портативни между
операционните системи(като изключим разбира се уеб приложенията, които
са портативни по подразбиране) и представаляват затворена
система. Промени в съдържанието, което предлагат тези програми могат
да правят само техните автори. Това определено не е духа на текущите
\textbf{"`социални"'} приложения, които са актуалният стандарт. В тях
потребителите са активни и са движещ фактор са успеха на
приложението. Едва ли е нужно да цитирам феномени като "`Twitter"',
"`Facebook"', "`MySpace"', "`Wikipedia"' и "`YouTube"'. 

Целта на тази дипломна работа е разработката на система за езиково
обучение, предназначена за нуждите на българските ученици и студенти(а
на всеки друг потребител, който се нуждае от подобно приложение),
която да бъде платформено независима(да работи на всяка една
операционна система) и която да предоставя на потребителите възможност
да допринасят към базата от познание на системата. Системата бе
наречена {\bf Spellbook}.

Платформата и езика \textbf{Java} веднага идват наум, когато целта е
разработката на платформено независимо приложение. Заради многото
предимства - \emph{отличен набор от инструменти, високо производителна и
надеждна виртуална машина, богат набор от библиотеки и наличието на
портативна библиотека за създаване на графични потребителски
интерфейси} избрах именно Java за реализацията на Spellbook.

В разработката на проекта са използвани, само и единствено,
най-модерните Java технологии - интегрираната среда за разработка
IntelliJ IDEA, инструментът за управление на проекти Maven 2, Java
Standard Edition 6, Java Persistence API 2, Subversion. Използваните
бази данни са две - вградена база H2 за локалните клиенти и MySQL за
централизирана база данни.

Важно е да се отбележи, че проектът не е просто дипломна работа, а
реално приложение, което вече си има потребителска база(макар и
скромна) и се развива постоянно. Важно и да се отбележи, че проектът
се разработва прозрачно, под формата на проект с отворен код,
лицензиран под GNU General Public License версия 3. Кодът му е
достъпен в интернет, приемат се идеи и кръпки от потребителите,
същестува и система за докладване на проблеми. Софтуерът се предоставя
под формата на пакети за популярните Линукс дистрибуции, инсталатор за
Уиндоус и платформено независим архив за всички останали операционни
системи. В близко бъдеще се очаква създаването и на пакет за OS X.

В първата глава на дипломната работа e обоснована нуждата от
приложението, разработено в нея, и е направен сравнителен анализ на
няколко конкурентни приложения. Във втората глава на дипломната работа
е направен общ преглед на технологиите, които бяха необходими за
изграждането на приложението. Третата глава описва самото му
проектиране, необходимите модули и структурата на базата данни,
използвана от системата. Четвъртата глава съдържа описание на
програмната реализация.  Ръководството за потребителя е представено в
последната пета глава. Дипломната работа завършва с кратко заключение,
което представлява кратък обзор на работата свършена досега и поглед
към бъдещето на системата. Изходният код на програмата и списъкът с
използваната литература се намират в приложението към дипломната
работа.

%%% Local Variables: 
%%% mode: latex
%%% TeX-master: "master"
%%% End: 
