\chapter{Програмна реализация}
\section{Избор и изграждане на базата данни}

Една система за езиково обучение очевидно се нуждае, най-малко, от
някакви речникови данни, на които тя да базира своите модули. Тези
данни трябва да бъдат съхраняване някъде. Много подобни системи
съхраняват данните си директно в текствови или двоични файлове. Това,
обаче, не е особено ефективно когато искате да правите произволен
достъп до данните или да манипулирате много от тях едновременно. 

Едно много по-ефективно решение е да се използва вградена релационна
база данни - подход използвам в много популярни приложение, като
Mozilla Firefox(SQLite), OpenOffice.org(HSQLDB) и т.н. В света на Java
приложенията най-популярните вградени бази данни са HSQLDB, H2 и Apache
Derby(изестна още като Java DB). И трите са реализирани изцяло на Java
и се интегрират отлично с Java приложения. В настоящата дипломна работа H2
Database беше предпочетена поради изключително високото си
бързодействие, малкият размер оперативна памет, която тя използва,
отличната си документация, чести обновления и леснота на употреба.

За да бъде тя достъпна в приложението трябва да се изпълнят следните
стъпки: 
\begin{itemize}
  \item h2.jar трябва да бъде добавен в клас пътя на приложението 
  \item за достъп се използва JDBC драйвера org.h2.Driver
  \item използва се URL за достъп до базата от вида
    jdbc:h2:/path/to/file
  \item ако базата не съществува тя ще бъде създадена първия път,
    когато се свържете към нея
\end{itemize} 
\section{Моделиране на базата данни}

Дизайнът на базата данни на приложението е изчистен, интуитивен и
минималистичен. Основните таблици, в които се съхраняват данните му са
само 5 - DICTIONARIES, DICTIONARY\_ENTRIES, STUDY\_SETS,
STUDY\_SET\_ENTRIES, RANK\_ENTRIES и EXAM\_SCORE\_ENTRIES. Всичките
съдържат няколко общи полета - уникален идентификатор(id), дата на
създаване(created), дата на промяна(modified). Общите полета в
таблиците са изразени в Java кода под формата на абстрактен базов
клас, който всичко други класове от бизнес модела разширяват. 

Таблицата DICTIONARIES съдържа следната информация - име на речник,
име на иконката на речника, двата езика, които го
описват(from\_language, to\_language). Добавянето на нов речник в
системата неминуемо минава през добавянето на нов ред в тази таблица. 

Таблицата DICTIONARY\_ENTRIES съдържа в себе си думите на
речниците. Нейните полета включват дума, превод и референция към
речника, който притежава думата. Следва се ествествената логика, че
един речник притежава много думи.

Таблицата STUDY\_SETS съдържа информация за същестуващите набори от
думи за изучаване. Думите от в един набор трябва да бъдат от
единствен речник, затова таблицата притежава референция към речник.

Таблицата STUDY\_SET\_ENTRIES съдържа думи на един набор, които по
същество са просто препратки към DICTIONARY\_ENTRIES. Освен това имаме
и референция към STUDY\_SET таблицата.

В таблицата RANK\_ENTRIES се съдържа рейтинга на думите в даден
език. Думите с висок рейтинг са често срещани, а тези с малък -
рядко. Както бе упомената и в предната глава, таблицата се използва от
изпитния модул за разделяне на думите по трудности, а данните в нея са
базирани на статистически анализ проведен върху различни литературни
творби.

Таблицата EXAM\_SCORE\_ENTRIES служи за съхранение на резултатите от
положените изпити. В нея се съхранява следната информация - използван
речник, име, брой верни отговори, брой грешни отговори и трудност на
изпита. 
\section{Изграждане на потребителският интерфейс}

Приложението има графичен потребителски интерфейс, реализиран
посредством библиотеката Swing на Java. Присъстват типичните за
повечето приложения елементи като лента с менюта (menubar) и лента с
бутони за бърз достъп до фунции(toolbar). Освен това приложението се
интегрира със областта за нотификации на операционната система.

Особено голяма внимание беше отделено на създаването на аткрактивен и
интуитивен дизайн на потребителският интерфейс. Той бе преработвам
изцяло няколко пъти преди да достигне до текущия си вид. За
допълнително удобство на потребителите интерфейсът е достъпен на
български и на английски език(посредством Translator инфраструктурата
на проекта), като е много лесно добавянето на още езици. 

Swing поддържа концепцията на включваеми външни видове(pluggable look
\& feels) - те позволяват на едно Swing приложение да изглежда по
много различни начини - например като Windows приложение, GTK+
приложение и са начин да се стесни пропастта между Swing апликациите и
така наречените native апликации. Spellbook използва включваемите
външни видове и позволява на потребителя да избери този, който му
допада най-много.
\section{Обобщена архитектура на приложението}
\section{Модули на приложението}
\subsection{Ядро}
\subsection{Помощни пособия}
\subsection{Swing компоненти}
\subsection{Потребителски интерфейс}

%%% Local Variables: 
%%% mode: latex
%%% TeX-master: "master"
%%% End: 
