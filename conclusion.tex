\chapter*{Заключение}
\addcontentsline{toc}{chapter}{Заключение}
За всяко успеха на всяко приложение най-важна е комуникацията между
разработчиците и неговите потребители. В крайна сметка никой не знае
какво е най-необходимо в един проект повече от хората, които го
ползват. Добрата организация на комуникацията между програмисти и
потребители в системата за езиково обучение е една от най-важните и
характеристики. Възможността на потребителите активно да редактират и
разширавят речниковата база, както и да я синхронизират с централната
такава е иновативна в сферата.

Другите и отличителни качества са използването на съвременни
технологии и методологии по време на разработката, както и
платформената независимост на продукта. Проектът има високо ниво на
модулярност и части от него могат лесно да бъдат преизползвани в други
проекти, ако някой прояви желание да го направи. Spellbook затвърждава
теорията, че Java е логичният избор за разработване на платформено
независими десктоп приложения, благодарение на нейната отлична
библиотека за разработка на потребителски интерфейси Swing.

В никой проект, разбира се, не може всичко да стане идеално и винаги
има възможност нещата да бъдат направено по-гъвкаво и по-добре. Една
от идеите за бъдещето на Spellbook е да се премине към използването на
OSGi контейнер, който да подобри още модулярността на проекта и да
позволи динамичното зареждане на части от него - на практика
компоненти като изпитът, иструментът за проверка на правописа и
т.н. биха могли да бъдат реализирани на "`разширения"' и потребителите
да ги активират само ако се нуждаят от тях. Друга интересна идея е за
някои платформи, като GNU/Linux и Windows да се добавят някои
възможности, които Java не поддържа директно, но са възможни
посредством JNI(Java Native Interface) - например нотификации за
преводите във FreeDesktop балончета под GNU/Linux, вместо в системния
трей. Или пък реализира на глобални клавиши - команди в Spellbook
достъпни, където и да се намирате в момента във вашата операционна
система. 

В предходните глави на тази дипломна работа вече бяха изложени обзор
на използваните технологии, цикъла на проектиране и реализиране на
системата, както и ръководството за потребителя. В приложението, което
следва, има списък на използваната литература, както и разпечатка на
целия изходен код на системата за езиково обучение Spellbook.
%%% Local Variables: 
%%% mode: latex
%%% TeX-master: "master"
%%% End: 
