\chapter{Проектиране на система за езиково обучение}
\section{Преглед на системата}
\section{Преглед на платформата и езика Java}
\section{Преглед на Swing}
\section{Преглед на H2}
\subsection{Обзор}
H2 е високо производителна релационна система за управление на бази
данни написана на Java. Тя може да се използва както в клиент-сървър
режим, така и във вграден режим. Размерът е много малък - само около
един 1МБ. H2 е свободен софтуер и се разпространява под модифицирана
версия на Mozilla Public License и под оригиналната версия на Eclipse
Public License. Тя се явява наследник на Hypersonic DB, макар, че не
споделя никакъв код с нея. H2 първоначално е означавало Hypersonic 2.

\subsection{Основни характеристики}
H2 поддържа поднабор на SQL(Structured Query Language). Основните
програмни интерфейси са SQL и JDBC, като освен това базата данни
поддържа използването на PostreSQL ODBC драйвер, който и позволява да
симулира PostgreSQL сървър.

Възможно е да се създават както таблици съхранявани в оперативната
памет(полезни за тестове), така и таблици съхранявани на твърдия
диск. Таблиците могат да бъдат постоянни или временни. Индекс типовете
са хеш таблица и дърво за таблиците съхванявани в паметта, и б-дървета
за таблиците съхванявани на твърдия диск. Всички операции,
манипулиращи данни, са транзакционални. Реализирани са заключване на
ниво таблици и многоверсиен контрол на паралелизма(multiversion
concurrency control). Двуфазният протокол за съхранение на данни се
поддържа също, но H2 не предлага имплементация на стандартен интерфейс
за разпределени транзакции. Що се отнася до сигурността H2 предлага
следните характеристики: права за достъп, базирани на роли, криптиране
на паролата с алгоритъм SHA-256, криптиране на данните с AES или Tiny
Encryption Algorithm, XTEA. Криптографските възможности са достъпни и
в самата база данни, като функиции. SSL/TSL връзки се поддържат от
клиент/сървър режима, както и когато се използва конзолното
приложение.

H2 включва и две имплементации на пълно текстово търсене - една H2
специфична и една използваща Lucene. H2 включва и проста форма на
висока надеждност - когато се използва в клиент/сървър режим, базата
данни поддържа горещо прехвърляне(известно още като клъстеризация). H2
поддръжа защита от SQL инжекции посредством налагането на употребата
на параметризирани заявки.  
\section{Преглед на MySQL}
\section{Преглед на Hibernate}
\section{Преглед на Maven}
\subsection{Какво е Maven?} 
Отговорът на този въпрос зависи от собствената ви
гледна точка. По-голямата част от потребителите Maven ще нарекат Maven
"инструмент за изграждане": инструмент, използван за изграждане на
разгръщащи се артефакти от изходния код. Билд инженери и мениджъри на
проекти може да кажат за Maven като нещо по-всеобхватно: инструмент за
управление на проекта. Каква е разликата? Инструмент като Ant е
насочена единствено на предварителна обработка, събирането,
опаковането, тестване и разпространение. Инструмент за управление на
проекти, като например Maven предоставя superset от функции намерени в
изграждането инструмент. В допълнение към предоставянето на изграждане
на способности, Maven да пускате отчети, генериране на уеб сайт, и да
се улесни комуникацията между членовете на работната група.

По-формално определение на Apache Maven: Maven е инструмент за
управление на проекти, които обхващат модели проект обект, един набор
от стандарти, на жизнения цикъл на проекта, система за управление на
зависимост и логика за изпълнение плъгин цели на определени етапи в
жизнения цикъл. При използване на Maven, описали вашия проект с
помощта на добре дефинирани модел проект обект, Maven могат да се
прилагат пресечни логика от поредица от общи (или обичай) плъгини.  Не
позволявайте на факта, че Maven е "Управление на проекти" инструмент,
който изплаши. Ако сте просто гледам за изграждане на инструмент,
Maven ще си свършат работата. Всъщност, първите глави на тази книга ще
се справят с най-широко използвани при: използване на Maven за
изграждане и разпространение на проекта.
\subsection{Конвенция вместо конфигурация}

%%% Local Variables: 
%%% mode: latex
%%% TeX-master: "master"
%%% End: 
