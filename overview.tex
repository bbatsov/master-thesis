\chapter{Проектиране на система за езиково обучение}
\section{Преглед на системата}
\subsection{Функционалност}
Всяка система за езиково обучение трябва да предлага минимум речников
модул и различни изпитни модули. Изпитните модули, обикновено, биват
изградени на думите налични в речниковата база на приложението и
допълнителна метаинформация за тях. 

В Spellbook речниковата база и метаинформацията, асоциирана с нея, се
съхраняват във вградена релационна база данни(H2). Това е голям
напредък от стандартното решение, езиковата база да се съхранява в
текстови файлове или в специално организирани двоични файлове по
няколко причини:
\begin{itemize}
  \item Имаме възможност за изключително ефективно търсене в базата -
    елиминира от последователно обхождане на записи във файл
  \item Имаме възможост да моделираме данните в речниковата база, по
    начин, който отразява отношенията между тях
  \item Имаме възможност лесно да манипулираме голям обем от данни
    едновременно
  \item Имаме възможност да използваме обектно-релационнен
    преобразувател като Hibernate, който да установи връзка между
    обектите в нашето приложение и данните в таблиците на базата данни.
\end{itemize}

Речниковия модул предоставя възможност за търсена на дефинициите на
думи в него. Търсенето може да е точно или приблизително, ако не бъде
открито точно съвпадение. Освен това речниковия модул е наясно с
връзките между наличните речници в речниковата база данни и поддържа
автоматично преключване между езиците в един двупосочен речник. За
удобство на потребителите речниковия модул трябва да поддържа
интеграция с клипборда на система и история на търсените досега думи.

Стандартни изпитни модули в една система са превод на думи от един
език на друг и преговор на нови думи. Диктовките са сериозна част от
езиковата подготовка също. Модул за проверка на правописа обикновено служи
за тяхната проверка - ученикът просто копира текста, който той е
написал и моментално вижда броя на допуснатите от него грешки, както и
верните думи. В един по-комплексен сценарий посредством интеграции със
сканиращо устройство и програма за разпознаване на текст може
ръкописни диктовки да бъдат дигитализирани от учениците и
преподавателите и проверени автоматично от системата. 

Научно доказано, е че човек(особено децата) усвояват материала, когато
той е поднесен забавно. Затова е от голямо значение някои от модулите
на системата да са организирани във вид на образователни игри.

Системата трябва да е изключително лесна за инсталиране и да не
изисква никаква начална конфигурация от своите
потребители. Настройките, зададени по подразбиране трябва да отговарят
на нуждите на голямата част от потребителите. В същото време системата
трябва да предлага високо ниво на гъвкавост - всички нейни по-ключови
аспекти трябва да се конфигурируеми. 

Системата трябва да е лесна за разширяване - добавянето на нови езици
в нея трябва да става по-максимално прост начин. Освен това системата
трябва да дава възможност на потребителите си да я обогатят като
коригират грешки в речниковия фонд, разширяват речниковия фонд и
докладват грешки в системата и предложения за подобрения и нови
изпитни модули.

В ерата на Интернет и уеб базираните приложения е важно една такава
система да предлага както десктоп вариант, така и уеб вариант, за да
удовлетворява нуждите и предпочитанията на възможно най-голям брой
потребители. 
\subsection{Езикова база данни}
Езиковата база данни е в основата на всички модули на системата. От
ключово значение е да бъде избран максимално добрия формат за
съхранението на данните, който да гарантира бърз достъп до тях и
възможност за лесно разширяване на базата. 

Вградените релационните системи за упревление на бази данни са
идеалното решение за такива ситуации. Те предлагам висока
производителност, малко потребеление на памет и голяма
гъвкавост. 

Езиковата база данни се състоия от речников фонд и метаинформация за
думите(например честотата на срещането им в езика, правила за
образуване на множествено число) и езика(азбука, граматически
правила).

Подходящата метаинформация може да бъде използвана от различни модули
- например честотата на думите е един език може да бъде използвана от
изпитен модул за да разделя думите по трудност(по-рядко срещаните думи
се считат за по-трудни), а модула за проверка на правописа би
използвал честотата за генериране на списък с най-вероятните думи,
вместо сбърканата от вас.
\subsection{Комуникация с потребителите}
Потребителя винаги е прав и винаги знае от какво се нуждае системата,
тъй като реално той е този, който я използва. Разработчиците винаги
трябва да поддържат добро ниво на комуникация със своите потребители и
да се вслушват в техните критики и предложения. 

Системата представена в тази дипломна работа разполага с уеб сайт, на
който потребителите могат да маркират открити грешки и да споделят
предложения. Освен това системата разполага и с дискусионен списък,
където се обсъждат новите възможности на системата, цикъла на
разработка и други интересни неща от "`кухнята"' и. 

Интересни възможности са също така функционалност за автоматично
докладване на грешки, известяване за нови версии и т.н.
\subsection{Разпространение}
Всеки софтуерен продукт трябва евентуално да стигне до своите
потребители. Начините, по които това може да стане са разнообразни -
официалния сайт на проекта, алтернативни ftp и http сървъри, торент
тракери и т.н. 

Важно е да се обръща внимание на спецификата на операционната система
на потребителите, за да бъде инсталирането на софтуера максимално
удобно и интуитивно за тях. Въпреки, че софтуера написан на Java е
платформено независимим не трябва да забравяме някои неща:
\begin{itemize}
  \item За Windows потребителите е желателно да бъде предоставен
  инсталатор
  \item За потребителите на различните GNU/Linux дистрибуции е добре
    да бъдат предоставени специфичните пакети - например RPM за Red
    Hat базирани дистрибуции, deb за Debian базирани дистрибуции и
    т.н.
  \item За потребителите на OSX също трябва да бъде предоставена
    специална версия, лесна за инсталиране там
  \item За всички останали потребилите трябва да се предостави
    платформено архив, лесен за използване независимо от операционната система 
\end{itemize}
\section{Преглед на платформата и езика Java}
\subsection{Java като платформа за програмиране}
Първото издание на Java през 1996 г., генерира невероятна
възбуда, не само в компютърната преса, но в масовите медии като "`Ню
Йорк Таймс"', "`Вашингтон пост"', както и "`Седмица бизнес"'. Java има
честта да бъде  първият и единствен език за програмиране, на когото е
отделена десет минутна история по Националното обществено радио. Сто
милиона щатски долара фонд за рисков капитал е създаден единствено за
продукти, произведени с използването на специфичен компютърен език.

Java безспорно е един добър език за програмиране, но неговата истинска
сила идва от Java платформата, която го заобикаля. Java никога не е
бил само език за програмиране. Има купища езици за програмиране, но
много малко от тях правят такова впечатление. Java е цяла платформа с
огромна библиотека, съдържаща много преизползваем код, с
инфраструктура за изпълнение, която предоставя сигурност, преносимост
между различните операционни системи и автоматично събиране на
боклука.

Програмистите желаят език с приятен и интуитивен синтаксис и
разбираема семантика. Java отговаря на това описание, но на него
отговарят още една дузина програмни езици. Някои от тях предлагат и
портативност и автоматично събиране на боклука, но нямат кой знае
каква стандартна библиотека и ви карат да правите много неща отначало
сами. Java ви предлага всичко - добър език, невероятна платформа за
изпълнение и огромна библиотека. Комбинацията от тези характеристики
прави Java неустоим избор за много програмисти по света и бе голям
фактор за избора на Java за реализацията на тази дипломна работа. 
\subsection{Ключови характеристики на Java}
Авторите на Java изтъкват 11 ключови характеристики на езика. В
следващите няколко реда съм изложил техните думи и кратък коментар:
\begin{itemize}
\item \textbf{Прост}
 
  \emph{Искахме да изградим система, която да може да бъде програмирана
    лесно без много допълнително обучение и която използва настоящите
    утвърдени практики. Въпреки, че С++ не отговаря на тази критерии,
    ние базирахме Java на него доколкото бе възможно за да направим
    системата по-лесна за разбиране. Java пропуска повечето рядко
    използвани, криво разбрани и объркващи възможности на C++, които
    смятаме, че носят повече проблеми, отколкото дивиденти.}

  Синтаксисът на Java наистина е изчистена версия на синтаксисът на
  C++. Няма нужда от заглавни файлове, указателна аритметика(или дори
  указателен синтаксис), структури, обединения, презареждане на
  оператори, виртуални базови класове и т.н. Дизайнерите на Java,
  обаче, не са се опитали да оправят всички проблемни неща в С++ -
  синтаксисът на твърдението switch например остава непроменен. Всеки,
  който владее С++ няма да има особени проблеми в прехода към Java.
\item \textbf{Обектно ориентиран}

    \emph{Казано просто, обектно-ориентирания дизайн е техника за
    програмиране, при която фокуса е върху данните(обектите) и върху
    интерфейсите към този обект. Ако направим аналогия с
    дърводелството - един "`обектно-ориентиран"' дърводелец ще го е
    грижа повече за стола, който прави, като иструментите, с които
    работи са вторични. Един "`не-обектно-ориентиран"' дърводелец би
    мислил главно за инструментите си. Обектно-ориентираните пособия
    на Java са по същество същите като в С++.}

  Обектно ориентираното програмиране е доказало своите качества през
  последните 30 години и почти немислимо модерен програмен език да не
  го използва(изключение, разбира се, са модерните функционални езици
  за програмиране). Наистина обектно-ориентираните характеристики на
  Java са сравними с тези на С++. Единствената по-голяма разлика, е че
  в Java множественото наследяване е заменено с по-простата концепция
  на интерфейси. Друга разлика е метаклас модела на Java.
  \item \textbf{Мрежов}

    \emph{Java има богата библиотека от процедури за работа с TCP/IP
    протоколи като HTTP и FTP. Java приложенията могат да отварят и
    достъпват обекти през Мрежата чрез URL със същата лекота, с която
    достъпват локалната файлова система.}

    Мрежовите възможности на Java са наистина големи и лесни за
    употреба. Всеки, който се опитал да програмира за Интернет с някой
    друг език ще признае колко лесни са в Java сложни задачи като
    отване на сокет връзка. Отдалеченото изпълняване на методи пък
    позволява комуникация между разпределени обектни.

  \item \textbf{Надежден}

    \emph{Java е предназначена за създаване на програми, които трябва да
    бъдат надеждни по различни параграфи. Java отдава голямо значение
    на ранното проверяване за потенциални проблеми, по-късното
    динамично(по време на изпълнение) проверяване и елеминирането на
    ситуации, които предразполагат към грешки. Най-голямата разлика
    между Java и C/C++ е, че Java има указателен модел, който
    елиминира възможността за презаписване на памет и повреждане на
    данни.}

  Тоза характеристика определено е много полезна. Java компилатора
  засича много проблеми, които в други езици биха се появили само по
  време на изпълнение на програмата. Колкото до втората точка от
  цитата - всеки, прекарал часове преследвайки проблем с паметта
  причинен от указател, ще е много щастлив в Java, поради
  невъзможността за възникване на подобни ситуации.

\item \textbf{Сигурен} 

  \emph{Java е предназначен за използване в мрежови/разпределени среди. В
  тази връзка, голямо внимание бе обърнато на сигурността. Java
  позволява създаването на системи без вируси, които е трудно да бъдат
  повлияни от недоброжелатели.}

От самото начало, Java е създадена да направи някои видове атаки
невъзможни. Сред тях са:

\begin{itemize}
\item Преливане на стека на изпълнение
\item Повреждане на паметта извън собственото адресно пространство
\item Четене и запис на файлове без необходитемите права
\end{itemize} 

В последствия в Java биват добавени още възможности в областтан на
сигурността - например в Java 1.1 е добавена концепцията за цифрово
подписани класове, която дава сигурност кой точно е автора на даден клас.

  \item \textbf{Архитектурно неутрален}
    \emph{Компилаторът генерира архитектурно-неутрален обектен файлове
    формат. Компилираният код е изпълним на много процесори, при
    наличието на Java среда за изпълнение. Java компилатора постига
    това посредством генерирането на байткод инструкции, които нямат
    нищо общо с конкретната компютърна архитектура. Те са проектирани
    да бъдат лесни за интерпретиране и лесни за превеждане в машинен
    код по време на самото изпълнение.}

    Това определено не е нова идея - преди повече от 30 години Никлас
    Вирт използва този подход, когато създава оригиналната
    имплементация на Паскал. 

    Разбира се, интепретирането на байткод е по-бавно от изпълнението
    на машинни инструкции. За щастие, виртуалните машини имат
    способността да превеждат най-често използваните инструкции в
    машинен код, процес известен като "`компилация точно
    навреме"'(just in time). Тази стратегия се оказва толкова
    ефективна, че дори платформата .NET на Microsoft я
    използва. Използването на виртуална машина дава и други
    предимства. То повишава нивото на сигурност, тъй като виртуалната
    машина може да проверява поведението на последователност от
    инструкции. Някои програми дори произвеждат байткод динамично, по
    този начин разширавяйки възможностите на работеща в момента програма.

  \item \textbf{Портативен}

    \emph{За разлика от C и С++, няма "`зависи от имплементацията"' аспекти на
      спецификацията. Размерите на примитивните типове данни са
      специфицирани, като и поведението на аритметиката с тях.}

    В Java int винаги е 32 битово цяло число. В C/C++ int може да означава
    16 битово цяло число, 32 битово цяло число, или друг размер, който е
    бил предпочетен от доставчика на компилатора. Единствената рестрикция,
    е че типа int трябва да има поне толкова байтове, колкото short int и
    че не може да има повече байтове от long int. Наличието на фиксиран
    размер за числените типове елиминира огромни главоболия при
    пренасянето на едно от приложение от една платформа на
    друга. Двоичните данни се съхраняват и пренасят във фиксиран формат,
    елиминирайки проблеми с подреждането на байтовете. Низовете се
    съхраняват в стандартен Unicode формат.

    \emph{Библиотеките, които са част от системата определят портативните
    интерфейси. Например имаме абстрактен клас Window и имплементации за
    него за UNI, Windows и Macintosh.}

    Всеки, който го е опитвал, знае че са необходими усилия с героични
    пропорции да се напише програма, която изглежда добре на Windows,
    Macintosh и десет разновидности на UNIX. Java 1.0 направи това
    героично усилие, като достави набор от инструменти, който поддържаше
    елементи на потребителския интерфейс общи за няколко платформи. За
    съжаление, резултатът беше библиотека, която с много работа донасяше
    резултати, които бяха едва приемливи на различните системи. Но това
    беше някакво начало - има много приложения, за които преносимостта е
    по-ценна от добре изглеждащия потребителски интерфейс и тези
    приложения имаха полза дори от ранните версии на Java. В момента
    библиотеката за графични потребителски интерфейси е изцяло нова,
    написана отначало и вече не разчита на компонентите на
    системата-домакин. Резултата е много по-еднороден и много
    по-атрактивен от това, което ни предлагаха ранните версии на Java. 

  \item \textbf{Интепретиран}

    \emph{Java интерпретаторът може да изпълнява Java байткод директно на всяка
    една машина, за която същестува версия на интерпретатора. Тъй като
    свързването е по-инкрементален и лек процес, цялостния цикъл на
    разработка може да бъде по-бърз.}

    Инкременталното сврързване има предимства, но неговите предимства за
    процеса на разработка определено са по-скромни, отколкото твърдят
    авторите на Java. Ранните версии на Java инструментите за разработка
    бяха действително доста бавни. Днес, обаче, байткода се транслира в
    машинен код от JIT компилатор.
  \item \textbf{Високопроизводителен}

    \emph{Въпреки, че производителността на интерпретирания байткод е
    обикновено повече от адекватна, има ситуации, в които е необходима
    по-висока производителност. Байткод може да бъде транслиран в
    машинен код за сътвения процесор, върху който работи приложението,
    по време на изпълнение.}

    В ранните години на Java много потребители възразяваха с
    твърдението, че производителността е "`повече от
    адекватна"'. Днес, обаче, повечето JIT компилатори са толкова
    добри, че могат да се съзтезават успешно с традиционните
    компилатори и в някои ситуации, дори да предоставят по-голяма
    производителност, тъй като разполатат с повече
    информация. Например JIT компилатор може да следи кои сегменти от
    кода се изпълняват най-често и да ги оптимизира
    допълнително. По-комплексен пример е елиминирането на извиквания
    на методи(процес известен като вграждане). JIT компилатора знае
    кои класове са заредени. Той може да използва вграждане, когато
    базирано на заредените в момента класове определен метод не е
    презаписан и може да махне оптимизацията по-късно, ако това е необходимо.  


  \item \textbf{Многонишков}

       \emph{Предимствата на многонишковите програми са по-добра
       производителност и време за реакция.}

       Хората, писали многонишков код в други езици, вероятно ще
       останат приятно изненадани от това колко лесно става това в
       Java. Нишките в Java позволяват да се разгърне цялата мощ на
       многопроцесорна система, ако това се поддържа от операционна
       система. Негатив е, че нишковите имплементации на различните
       операционни системи се различават много и Java не се стреми да
       бъде платформено независима в тази насока. Само кода за
       извикване на многонишкови функции е един и също между
       различните платформи; Java делегира имплементацията на нишките
       на операционната система или на нишкова библиотека. Все пак
       леснотата, с която се работи с много нишки в Java е една от
       ключовите му черти и една от основните причини да е
       предпочитана платформа за разработка на сървърни приложения.


  \item \textbf{Динамичен}

    \emph{В много отношения Java е по-динамичен език от С и С++. Той е
    проектиран да се адаптира към развиващи се среди. Библиотеките
    могат да добавят спокойно методи и полета без никакви ефекти върху
    техните клиенти. В Java намирането на информация за тип по време
    на изпълнението е просто.}

    Това е важна характеристика в случаите, когато код трябва да бъде
    добавен към работеща програма. Такъв пример е код, който се тегли
    от интернет и работи в браузър. В Java 1.0 намирането на типова
    информация по време на изпълнение определено не беше лесна
    задача. Текущите версии на Java, обаче, дават на програмистите
    достъп до структурата и поведението на обектите. Това е много
    полезно за приложения, които трябва да анализират обекти по време
    на изпълнение - дизайнери на графични интерфейси, дебъгери,
    включваеми компоненти и обектни бази данни.

\end{itemize}
\section{Преглед на Swing}
\section{Преглед на H2}
\subsection{Обзор}
H2 е високо производителна релационна система за управление на бази
данни написана на Java. Тя може да се използва както в клиент-сървър
режим, така и във вграден режим. Размерът е много малък - само около
един 1МБ. H2 е свободен софтуер и се разпространява под модифицирана
версия на Mozilla Public License и под оригиналната версия на Eclipse
Public License. Тя се явява наследник на Hypersonic DB, макар, че не
споделя никакъв код с нея. H2 първоначално е означавало Hypersonic 2.

\subsection{Основни характеристики}
H2 поддържа поднабор на SQL(Structured Query Language). Основните
програмни интерфейси са SQL и JDBC, като освен това базата данни
поддържа използването на PostreSQL ODBC драйвер, който и позволява да
симулира PostgreSQL сървър.

Възможно е да се създават както таблици съхранявани в оперативната
памет(полезни за тестове), така и таблици съхранявани на твърдия
диск. Таблиците могат да бъдат постоянни или временни. Индекс типовете
са хеш таблица и дърво за таблиците съхванявани в паметта, и б-дървета
за таблиците съхванявани на твърдия диск. Всички операции,
манипулиращи данни, са транзакционални. Реализирани са заключване на
ниво таблици и многоверсиен контрол на паралелизма(multiversion
concurrency control). Двуфазният протокол за съхранение на данни се
поддържа също, но H2 не предлага имплементация на стандартен интерфейс
за разпределени транзакции. Що се отнася до сигурността H2 предлага
следните характеристики: права за достъп, базирани на роли, криптиране
на паролата с алгоритъм SHA-256, криптиране на данните с AES или Tiny
Encryption Algorithm, XTEA. Криптографските възможности са достъпни и
в самата база данни, като функиции. SSL/TSL връзки се поддържат от
клиент/сървър режима, както и когато се използва конзолното
приложение.

H2 включва и две имплементации на пълно текстово търсене - една H2
специфична и една използваща Lucene. H2 включва и проста форма на
висока надеждност - когато се използва в клиент/сървър режим, базата
данни поддържа горещо прехвърляне(известно още като клъстеризация). H2
поддръжа защита от SQL инжекции посредством налагането на употребата
на параметризирани заявки.  
\section{Преглед на MySQL}
\section{Преглед на Hibernate}
\section{Преглед на Maven}
\subsection{Какво е Maven?} 
Отговорът на този въпрос зависи от собствената ви
гледна точка. По-голямата част от потребителите Maven ще нарекат Maven
"инструмент за изграждане": инструмент, използван за изграждане на
разгръщащи се артефакти от изходния код. Билд инженери и мениджъри на
проекти може да кажат за Maven като нещо по-всеобхватно: инструмент за
управление на проекта. Каква е разликата? Инструмент като Ant е
насочена единствено на предварителна обработка, събирането,
опаковането, тестване и разпространение. Инструмент за управление на
проекти, като например Maven предоставя superset от функции намерени в
изграждането инструмент. В допълнение към предоставянето на изграждане
на способности, Maven да пускате отчети, генериране на уеб сайт, и да
се улесни комуникацията между членовете на работната група.

По-формално определение на Apache Maven: Maven е инструмент за
управление на проекти, които обхващат модели проект обект, един набор
от стандарти, на жизнения цикъл на проекта, система за управление на
зависимост и логика за изпълнение плъгин цели на определени етапи в
жизнения цикъл. При използване на Maven, описали вашия проект с
помощта на добре дефинирани модел проект обект, Maven могат да се
прилагат пресечни логика от поредица от общи (или обичай) плъгини.  Не
позволявайте на факта, че Maven е "Управление на проекти" инструмент,
който изплаши. Ако сте просто гледам за изграждане на инструмент,
Maven ще си свършат работата. Всъщност, първите глави на тази книга ще
се справят с най-широко използвани при: използване на Maven за
изграждане и разпространение на проекта.
\subsection{Конвенция вместо конфигурация}

%%% Local Variables: 
%%% mode: latex
%%% TeX-master: "master"
%%% End: 
